\section{Matrix copy via block reverse ordering}
\textit{Task1} and \textit{Task2} ask for two functions, \texttt{routine1()} and \texttt{routine2()} respectively, that %
take a matrix $M$ of size $N\times N$ and reverse the order in blocks $b\times b$ in another matrix $O$.\\%
As the previous exercise, the only difference between the two functions is whether implicit parallelism is used. I chose %
again to avoid code duplication and accomplish what was requested via compilation flags, the same used in the previous %
exercise. In addition, the \textit{bash} script was integrated with the new commands and, as before, each version was %
executed three times with $b$ values ranging from $2^2$ to $2^8$ (inclusive).\\%
Source code \ref{code:matrix} shows the implemented algorithm. It loops around each $b\times b$ block in the original %
matrix $M$ and, for each, copies the values of the current block and the opposite one in the reverse order. Doing both %
copies in the same iteration allows the main matrix to be looped only for half of the rows.
\begin{code}
    \captionof{listing}{\label{code:matrix}Implemented algorithm}
    \begin{minted}{c}
#include <time.h>

float **routine1(float **m, int dim, int b, float **o) {
    clock_t t0, t1;

    t0 = clock();
    for (int i = 0; i < dim / 2; i += b) {
        for (int j = 0; j < dim; j += b) {
            for (int k = 0; k < b; k++) {
                for (int l = 0; l < b; l++) {
                    o[i + k][j + l] = m[dim - i - (b - k)][dim - j - (b - l)];
                    o[dim - i - (b - k)][dim - j - (b - l)] = m[i + k][j + l];
                }
            }
        }
    }
    t1 = clock();

    printf("%i,%12.4f\n", b, (t1 - t0) / 1000000.0);
    return (float **) o;
}
    \end{minted}
\end{code}


\subsection*{Results analysis}
Table \ref{table:matrix_seq} and Table \ref{table:matrix_par} show the run times obtained for each execution. The parallel %
version is 49\% (average) quicker than the same algorithm executed sequentialy.%

\begin{table}[h!tb]
    \centering
    \parbox{.45\linewidth}{
        \centering
\caption{\label{table:matrix_seq}Run times by $b$ size - Sequential (times in seconds)}
\begin{tabular}{@{} c c c c c @{}}
\toprule
    \textbf{Size} & \textbf{Run 1}& \textbf{Run 2}& \textbf{Run 3}& \textbf{Average}\\
\midrule
    $2^2$ & 0.1200 & 0.1300 & 0.1300 & 0.1267\\
\lightrule
    $2^3$ & 0.1400 & 0.1400 & 0.1200 & 0.1333\\
\lightrule
    $2^4$ & 0.2000 & 0.2000 & 0.2000 & 0.2000\\
\lightrule
    $2^5$ & 0.1900 & 0.1700 & 0.1800 & 0.1800\\
\lightrule
    $2^6$ & 0.1600 & 0.1600 & 0.1600 & 0.1600\\
\lightrule
    $2^7$ & 0.1500 & 0.1500 & 0.1500 & 0.1500\\
\lightrule
    $2^8$ & 0.1400 & 0.1300 & 0.1300 & 0.1333\\
\bottomrule
\end{tabular}
%
    }
    \parbox{.50\linewidth}{
        \centering
\caption{\label{table:matrix_par}Run times by $b$ size - Parallelised (times in seconds)}
\begin{tabular}{@{} c c c c c @{}}
\toprule
    \textbf{Size} & \textbf{Run 1}& \textbf{Run 2}& \textbf{Run 3}& \textbf{Average}\\
\midrule
    $2^2$ & 0.0700 & 0.0800 & 0.0800 & 0.0767\\
\lightrule
    $2^3$ & 0.0700 & 0.0600 & 0.0700 & 0.0667\\
\lightrule
    $2^4$ & 0.0800 & 0.1000 & 0.0900 & 0.0900\\
\lightrule
    $2^5$ & 0.0800 & 0.0700 & 0.0700 & 0.0733\\
\lightrule
    $2^6$ & 0.0800 & 0.0900 & 0.0800 & 0.0833\\
\lightrule
    $2^7$ & 0.0700 & 0.0900 & 0.0800 & 0.0800\\
\lightrule
    $2^8$ & 0.0700 & 0.0700 & 0.0700 & 0.0700\\
\bottomrule
\end{tabular}
%
    }
\end{table}

To compute the effective bandwidth, it is necessary to know the number of bytes read $B_r$ and written $B_w$. In our %
case, both values are equal and calculated as $(N\times N) * S_f$, where $N\times N$ is the size of the matrix $M$ and %
$S_f$ is the size of a \textit{float} type (4 bytes). The algorithm reads two values and stores the same %
quantity, totaling four operations for each matrix position. Substituting this information in equation (\ref{eq:bandwidth}), %
we get the results in Table \ref{table:bandwidths}. Plot \ref{plot:matrix} shows a visual representation of the effective %
bandwidth.

\begin{equation}
    \label{eq:bandwidth}
    b=\frac{(B_r+B_w)/10^9}{t}=\frac{(4*4096^2*4)/10^9}{t}\qquad[\textnormal{GB/s}]
\end{equation}

\begin{table}[h!tb]
    \centering
    \caption{\label{table:bandwidths}Effective bandwidths in bytes/s}
    \begin{tabular}{@{} c c c @{}}
    \toprule
        \textbf{Size} & \textbf{Sequential}& \textbf{Parallelised}\\
    \midrule
        $2^2$ & 2.12 & 3.50\\
    \lightrule
        $2^3$ & 2.01 & 3.96\\
    \lightrule
        $2^4$ & 1.34 & 2.98\\
    \lightrule
        $2^5$ & 1.49 & 3.66\\
    \lightrule
        $2^6$ & 1.76 & 3.22\\
    \lightrule
        $2^7$ & 1.79 & 3.36\\
    \lightrule
        $2^8$ & 2.01 & 3.83 \\
    \midrule
        Average & 1.79 & 3.50\\
    \bottomrule
    \end{tabular}
\end{table}
\begin{code}
    \captionof{listing}{\label{code:matrix}Implemented algorithm}
    \begin{minted}{c}
#include <time.h>

float **routine1(float **m, int dim, int b, float **o) {
    clock_t t0, t1;

    t0 = clock();
    for (int i = 0; i < dim / 2; i += b) {
        for (int j = 0; j < dim; j += b) {
            for (int k = 0; k < b; k++) {
                for (int l = 0; l < b; l++) {
                    o[i + k][j + l] = m[dim - i - (b - k)][dim - j - (b - l)];
                    o[dim - i - (b - k)][dim - j - (b - l)] = m[i + k][j + l];
                }
            }
        }
    }
    t1 = clock();

    printf("%i,%12.4f\n", b, (t1 - t0) / 1000000.0);
    return (float **) o;
}
    \end{minted}
\end{code}
